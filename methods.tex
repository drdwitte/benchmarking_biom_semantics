%\todo{Deze intro moet gereduceerd worden tot 1 lijn}

In this section we describe our benchmark methodology: we define a benchmark space and describe how it was explored. We explain how we made a pre-selection of triple stores
and how they were configured. We describe the datasets and queries used in this work. Finally we list our efforts at making the benchmark more easily reproducible.

%In this benchmark we would like to discover trends when modifying certain aspects of the benchmark setup. In the first subsection %~\ref{subsec:bmexplore} 
%we define a \emph{benchmark space}. For every dimension in this space we try to test at test two possible values.
%Performance is not always the first concern in a system architecture. In subsection on `Store Preselection' %~\ref{subsec:featurematrix} 
%we describe an approach using a \emph{Feature Matrix} in which weights can be assigned to certain properties of a system, in order to make a ranking of the different systems, given a use case. 
%The next subsection
%%~\ref{subsec:bmscheme} 
%gives a detailed explanation on our attempt at making the benchmark itself more easily reproducible and comparable with other work. In the final subsection %~\ref{subsec:dataqueries} 
%the benchmark data and query-sets are introduced. 

\subsection{Benchmark Space Exploration}
\label{subsec:bmexplore}

Following parameters are assessed:

\begin{itemize}
	\item \textbf{The choice of database engine:} We assess 7 different systems, 4 \emph{Vendors} and 3 \emph{Prototypes}. For the \emph{Vendors} we considered Blazegraph, GraphDB, Virtuoso and a non-disclosed Enterprise System (ES), which did not give permission to disclose its name. The \emph{Prototypes} are HDT~\cite{DBLP:journals/ws/FernandezMGPA13}, FedX~\cite{DBLP:conf/semweb/SchwarteHHSS11}, and Triple Pattern Fragments~\cite{DBLP:conf/semweb/VerborghHMHVSCCMW14}.
	\item \textbf{The server memory:} We distinguish between 32GB and 64GB of RAM.
	\item \textbf{The size of the (optionally) distributed system:} We run tests for single and 3-node setups when supported by the RDF database. Federated systems are configured with $N+1$ nodes, with $N$ the number of slave nodes (1 or 3), and 1 federator node. To clarify: $N=3$ thus corresponds to 3 instances for \emph{Vendor} systems, while $N=3$ for federated setups requires $3+1$ instances. The choice for $N=3$ is related to the fact that for one of the systems only a 3-node configuration is available.
	\item \textbf{The query properties:} The WatDiv benchmark query-set contains BGP queries, while the Ontoforce dataset consists mainly of complex aggregation-based queries with many filters.
	\item \textbf{The number of dataset triples:} We run 3 datasets of WatDiv, with 10~million, 100~million, and 1~billion triples. The Ontoforce dataset contains 2.4~billion triples.
	\item \textbf{The way in which the RDF system is configured:} We test two configurations for WatDiv1000M, namely the \emph{Documented} and the \emph{RFI-optimized} configuration, which will be defined later in this section. 
	\item \textbf{The state of the system when the query is launched:} We distinguish between a single-threaded warm-up run and a multi-threaded stress test (5 clients). We also investigate whether caching effects play a role in the runtime behavior.
\end{itemize} 

\todo{We are here: 27/09 18u}\\

Testing every possible combination of parameters is very time and resource consuming and not necessarily the most informative. Therefore we opted for a greedy exploration of this space consisting of 51 2-phase  benchmarks (incl. re-runs), each with a warm-up and a consecutive stress test (see Table~\ref{bmspace}).

%wordt in result intros al verwerkt
%As a performance measure there are also a number of metrics to choose from. If we focus on runtime we have to distinguish between comparing the \emph{runtime of the median query} or its runtime distribution versus the \emph{total runtime} of a full query mix. The first is independent of extreme values while the latter is  heavily affected by the runtime on the slowest queries and a possible timeout parameter. 
%The median query runtime is important for interactive systems while the total runtime is important to assess a system's usability in an ETL context. To take into account different types of system architectures, hardware and differing licensing costs we opt to use \emph{benchmark cost} as a unification parameter.

% Please add the following required packages to your document preamble:
% \usepackage{multirow}
\begin{table}[htbp!]
	\centering
	\caption{Benchmark overview.}
	\label{bmspace}
	\scalebox{0.85}{
	\begin{tabular}{l|l|cccc}

		\hline
			
		\textbf{Systems} & \textbf{Setup} & \multicolumn{3}{l}{\textbf{WatDiv}} &   \multicolumn{1}{l}{\textbf{Onto-}} \\
		\; & \; & \textbf{10M} & \textbf{100M} & \textbf{1000M} & \textbf{force} \\

		\hline

		\multirow{3}{*}{\begin{tabular}[c]{@{}l@{}}Vendors \\ (4)\end{tabular}}    & 32             & \checkmark                                & \checkmark                                 & \checkmark                                  &                                    \\
		& 64             &                                  &                                   & \checkmark                                  &                                    \\
		& 64/Opt         &                                  &                                   & \checkmark                                  & \checkmark                                  \\ \hline
		\multirow{2}{*}{\begin{tabular}[c]{@{}l@{}}Multi-\\ Node (2)\end{tabular}} & 3 x 32         &                                  &                                   & \checkmark                                  &                                    \\
		& 3 x 64/Opt     &                                  &                                   &                                    & \checkmark                                  \\ \hline
		\multirow{2}{*}{\begin{tabular}[c]{@{}l@{}}Prototypes  (3)\end{tabular}} & 64  &   & \checkmark  & \checkmark  & \checkmark \\
		& 3 x 64 &  & \checkmark  & \checkmark  & \checkmark    \\ \hline                             
	\end{tabular}
	}
\end{table}
