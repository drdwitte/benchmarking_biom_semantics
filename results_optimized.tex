%
%RESULTS Rev1
%RUNTIME
%SUCCESS/ERROR

%Convergence of results? Not with memory, with configs though...
After contacting the vendors with our initial results one of the parties suggested to demonstrate the optimal operation of their database. 
This was formalized by sending out a Request-For-Information (RFI), specifying the benchmarks we were planning to run.
3 out of 4 vendors chose to participate in the RFI, which resulted in an \emph{Optimized} configuration. 

In Figure~\ref{fig:Fig02_WatdivVerticalScaling} the rightmost panel corresponds to running the benchmark with the \emph{Optimized} configuration.

%
%median/mean
%Bla_N1_64_W1000_Def     2.825023 	6.741764
%Gra_N1_64_W1000_Def    40.375218 	59.564284
%Vir_N1_64_W1000_Def     0.175917 	3.667321
%
%Bla_N1_64_W1000_Opt    0.739366 	1.945429
%Gra_N1_64_W1000_Opt    0.652474  	6.320693
%Vir_N1_64_W1000_Opt    0.167420 	4.054050
%
\begin{itemize}
	\item \textbf{Sensitivity to configuration:} \textbf{Vir1\_64} got no benefit from the RFI settings file. For 
	\textbf{Bla1\_64} the only improvement was to explicitly configure the timeout parameter on the server side. This avoids unnecessary overhead while the client is already disconnected. It leads to a speedup of approximately 3.5 for both runtime measurements. \textbf{Gra1\_64} has the highest sensitivity to proper configurations. The provided scripts ensure a speedup of 9.4 for the average runtime and a median runtime speedup of 62.
	\item \textbf{32\_Def to 64\_Opt:} Moving from the left panel to the right in Figure~\ref{fig:Fig02_WatdivVerticalScaling}, we clearly see results converging in the rightmost window with the \textbf{64\_Opt} measurements. \textbf{Bla1\_64} is the most efficient system for processing batch workloads with an average runtime of 1.95s per query, 4.05s and 6.32s for \textbf{Vir1\_64} and \textbf{Gra1\_64} respectively. In the query performance \textbf{Vir1\_64} has a median runtime of 0.17s where \textbf{Gra1\_64} and \textbf{Bla1\_64} have runtimes of 0.65s and 0.74s respectively.
%Bla_N1_32_W100_Def   0.191359		0.461707	=> x5
%Gra_N1_32_W100_Def   0.047812		0.417214 	=> x15
%Vir_N1_32_W100_Def   0.046707		0.301350	=> x15
	\item \textbf{Runtime vs Dataset Size:} Returning to section~\ref{subsec:bigdata} we can verify that the linear scaling behavior is largely restored, confirming our earlier hypothesis. Multiplication factors drop to 4.2 for Blazegraph, for Virtuoso and GraphDB mf $ \approx 15$.
\item \textbf{Timeouts \& Errors:} Apart from 5\% timeouts for \textbf{Gra1\_64\_Opt}, no query errors are observed with the \emph{Optimized} configurations.
\end{itemize}
%Bla_N1_64_W1000_Opt:	Success: 100.	Error: 0.0	Timeout: 0.0
%Gra_N1_64_W1000_Opt:	Success: 95.0	Error: 0.0	Timeout: 5.0
%Vir_N1_64_W1000_Opt:	Success: 100.	Error: 0.0	Timeout: 0.0

