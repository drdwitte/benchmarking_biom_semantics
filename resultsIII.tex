
%errors excluded
\todo{Moet naar III} \\
A major concern when comparing query runtimes between different engines is \emph{query completeness}. The current query event format, shown in Table~\ref{table:queryevents}, explicitly reports whether a query was solved correctly, meaning it has retrieved the complete set of results. In the sections `Results III' query completeness is the topic of of the first subsection.
%~\ref{subsec:completeness} 
To interpret the results in this section correctly, it is important to understand that queries, which have incomplete results for at least one benchmark, are completely discarded in the runtime comparisons.
\todo{TOT HIER}
\todo{Dit moet ook tot een paar zinnen gereduceerd worden!}
%waarom uitleggen
% in store preselection staat het volgende:
%The comparison with these SemWeb systems was an essential part of the research collaboration with Onto-
%force as their initial goal was to build their DISQOVER search interface on top of a federated querying system.
%The advantage of the latter is that their interface would then provide a live view on a continuously updating
%Life Sciences Linked Data cloud, removing the need for an ETL process.
The WatDiv benchmark can serve as an initial testing procedure when selecting an appropriate triple store for a certain use case. The ETL case for Ontoforce bears some similarity to the WatDiv benchmark. 
The Ontoforce benchmark consists of interactive federated queries which are extracted from the user logs of the DISQOVER product. These queries are currently solved by combining an ETL preprocessing step, which integrates the different Life Sciences datasets offline using Ontoforce's own central ontology. This ETL step bears a lot of similarity with the WatDiv benchmark as it consists of mostly BGP queries. The queries of the Ontoforce benchmark are the result of faceted browsing, whereby, in practice, the facet filters are performed by a distributed search system (SOLR), but their product can also run with a SPARQL-based back-end. In this section we evaluate the ability of \emph{Vendor} systems to work with these types of queries and therefore serve as an alternative to a search system.

%Runtimes - minder belangrijk, meer focus op waarom
%N1_64_Opt Ontoforce => for both ES and Virtuoso no response time <> runtime
%Blazegraph crashed
%Es 380848.276950983	380849.972397525				=> coincide
%GraphDB crashed
%23258.549703492	23261.613780134						=> coincide
The Ontoforce benchmark has a very challenging query set. Therefore the focus of section~\ref{sec:realworld} will be far less on query runtimes but more on trying to extract insights which are generalizable. In this benchmark run the response times consistently coincide with the query runtimes.  
%More challenging => meer errors => besproken in aparte sectie
In subsection title `Benchmark Error Analysis' %~\ref{subsec:erroranalysis} 
we give more detailed insights in the behavior of the different systems on the Ontoforce benchmark. We pay special attention to query failures and query completeness.
In the following subsection
%~\ref{subsec:dtrees} 
we try to automatically infer the reasons behind query success, failure, different error types, and slow versus fast running queries. This automation is achieved by making use of decision tree analysis which should circumvent bias introduced by human interpretation.
In the final section we compare the results of all benchmarks in this research using \emph{Benchmark Cost} as a unification parameter. This allows to make comparisons between setups which are very different in nature and see whether the trends into the benchmark results are consistent. This approach also takes into account the financial cost for data ingestion and the different licensing fees.

\subsection{Benchmark Error Analysis}
\label{subsec:erroranalysis}
%
%RESULTS Rev1: Notebook Rev1_02  

%exacte cijfers over timeouts en aantal error in data notebook
%Crosstabs (ALL warmup & stress):	Success	Unverified	Incomplete	Timeout Error Unknown (all = 7338)
%Bla_N1_64_Ont_Opt 					36		0			0			431		0	  6871 (94%) 	
%Es_N1_64_Ont_Def 					4246	0			4			1257	1830  ALL 	
%Gra_N1_64_Ont_Opt 					1583	0			0			2372	0	  3383 (46%)	
%Vir_N1_64_Ont_Opt 					7069	120			60			41		48		ALL 	
%Vir32 first test on AWS => 2946 errors! (0 results while consensus > 0) => 41%
%new Vir32 OK same results as Vir64
%Vir64 Incomplete = 6 x 10 queries (<1%) => 1 query incomplete (2^20) andere 0 results terwijl consensus > 0

%Vir3_64_0	many errors but most queries solved at least once => 1122 queries in stres test with at least one susccesful run!
%Vir_N3_64_Ont_Opt_0 				4072	38			27			19		3182		ALL 	
%Vir_N3_64_Ont_Opt_2 				3918	66			2718		0		636			ALL 	
%Reruns telkens maar 100 queries until crash

%TODO notebook contains more info on query correctness

%RESULTS ORIGINAL PAPER
%BMSurvival: 
%	[('Bla_N1_64_Ont_Opt', 55), ('Gra_N1_64_Ont_Opt', 2541), ('ES_N1_64_Ont_Def', 7335), ('Vir_N1_64_Ont_Opt', %7338)]
%ERROR frequency per query
% 						Fail_always	Fail_never 		Fail_sometimes 	Unknown 	thread_type
% 	Bla_N1_64_Ont_Opt 	18 			37 				0 				1168 		warmup
% 	ES_N1_64_Ont_Def 	499 		724 			0 				0 			warmup
% 	Gra_N1_64_Ont_Opt 	233 		988 			0 				2 			warmup
% 	Vir_N1_64_Ont_Opt 	14 			1209 			0 				0 			warmup
%
% 	Bla_N1_64_Ont_Opt 	0 			0 				0 				1223 		stress
% 	ES_N1_64_Ont_Def 	289 		238 			696 			0 			stress
% 	Gra_N1_64_Ont_Opt 	400 		313 			150 			360 		stress
% 	Vir_N1_64_Ont_Opt 	15 			1208 			0 				0 			stress
%GraphDB fails after completing 21\% of the stress test.
%Virtuoso is consistent and successful in 99\% of the queries, ES survived benchmark and is consistently successful for 59\% and 19\% of the queries during %warmup and stress run respectively.
 %[('Vir_N3_64_Ont_Opt_0', 4313), ('Vir_N3_64_Ont_Opt_1', 179), ('Vir_N3_64_Ont_Opt_2', 7338), ('Vir_N3_64_Ont_Opt_AWS1', 116), ('Vir_N3_64_Ont_Opt_AWS2', 224), ('Vir_N3_64_Ont_Opt_AWS3', 228)]



\paragraph{Error Frequencies}

The \emph{Prototypes} and \textbf{Vir*} have been tested on the Ontoforce benchmark for our SWAT4LS~\cite{dewitte_swat4ls_2016} contribution. Note that \textbf{TPF} systems do not currently support all SPARQL operators and could therefore not be run on this benchmark.  In Figure~\ref{fig:Fig09_FailuresOntoforceBM} we show the results for the \emph{Vendor} systems. Each simulation consists of a small bar, corresponding to the single-threaded warm-up run, and 5 concatenated bars corresponding to 5 threads in the stress test.
The Figure also shows that only Virtuoso simulations had a sufficiently wide benchmark survival interval to enable further analysis.

\begin{itemize}
	\item \textbf{Bla1\_64\_Opt:} One major difference with the results on the WatDiv benchmark is Blazegraph's inability to handle the complexity of the Ontoforce queries, resulting in very short benchmark survival interval: it contains only 55 queries, of which 18 are timeouts.
	
	\item \textbf{Gra1\_64\_Opt:} GraphDB also did not survive the entire benchmark, but managed to stay up for 21\% of the stress run. During the stress run it solved 40\% of the queries successfully, the other queries resulted in a timeout. For 38\% of the queries, at least one successful run is available in the stress run.

	\item \textbf{ES1\_64\_Def:} ES was definitely the least successful on the WatDiv benchmarks, but is the only store, apart from Virtuoso, for which the benchmark survival interval spans the entire benchmark. 58\% of the queries were executed successfully. The remainder consists of 25\% HTTP errors and 17\% timeouts. 
%%Vir_N1_64_Ont_Opt 	7069	120			60			41		48		ALL 		
	\item \textbf{Vir1\_*\_Opt:} Virtuoso is both consistent and successful on this benchmark with only 1\% of queries consistently failing, overall the success rate is 98\%. These failures correspond mainly to queries which contain \emph{property paths}. None of the other stores could handle these queries. 
	It should be noted that during the creation of the DISQOVER product, Virtuoso was frequently used as a back-end system, which partially implies a certain favorable bias in the Ontoforce results.The \textbf{Vir1\_32\_Opt} in the SWAT4LS~\cite{dewitte_swat4ls_2016} paper had 41\% incomplete queries. This re-run however, achieves the same figures as the 64GB run.
	
	\item \textbf{Vir3\_64\_Opt\_*:} The  \textbf{Vir3\_64\_Opt} setup was re-run multiple times, the different runs are identified with an additional sequence number 0-2:
	Although the success rate of \textbf{Vir3\_64\_Opt\_0} is only 55\%, 92\% of the queries are successfully executed at least once, which makes it possible to make runtime comparisons. \textbf{Vir3\_64\_Opt\_2} has far less reported errors. Post-processing revealed issues with query completeness (orange) for 37\% of the queries.
\end{itemize}


 
\paragraph{Query Correctness.}

Previously published results~\cite{dewitte_swat4ls_2016} had counter-intuitive runtime results:  \textbf{ Vir1\_32} and \textbf{Vir3\_64\_Opt\_2} executed much faster than \textbf{Vir1\_64}. 
Consequently, we studied the number of results per query:

 
\begin{itemize}
	\item \textbf{Inter-thread consistency:} As a first step we analyzed whether for each individual system the number of query results was consistent for each query-mix. 
	Without any exception this inter-thread consistency was confirmed.
	
	\item \textbf{Query consensus:} In the query event format, described in Table~\ref{table:queryevents}, one field indicates whether a query is correct or its result count incomplete. These values are obtained by creating a query consensus, with the following rules. If at least two separate \emph{Vendor} systems agreed on the number of results we assume this results is `correct', for 97.3\% this is the case. If only 1 engine can solve a query we label these as `uncertain'. Virtuoso solves 19 queries for which no consensus can be derived. For 13 queries none of the systems managed to generate a solution. 8 of these contain a property path operator, the other 5 have \sql{FILTER IN} operators containing large URI lists, such that the file size of the query is between 10 and 100 kb.
	
	\item \textbf{Count Queries:} Of the 19 `uncertain' queries solved by Virtuoso 15 are \sql{count} queries. However, upon inspection the \sql{count} operator was always part of a sub-query, so this result can not be disproven. The benchmark software only reports the number of results per query. We extended it to also download the actual results to be able to verify whether the \sql{count} queries are consistent between the stores. However, no inconsistencies were found there.
	
	\item \textbf{Incorrect Query results:} Some of the Virtuoso benchmarks have incorrect results. The typical pattern is that the query is executed $< 1s$ and generates 0 results. 1 query also had the query result limit $ = 2^{20}$. To get more insight into the context of incomplete queries we executed the \textbf{Vir3\_64} benchmark an additional 3 times. In these runs the incorrect query results were not observed, but, but the new benchmarks never made it to the stress test, with the best run having a benchmark survival interval with a length of 228 queries.
\end{itemize}








\subsection{Decision Tree Analysis of Query Features}
\label{subsec:dtrees}
\begin{table*}[htbp!]
	\centering
	\caption{Query Features and information on their range and correlations with other (discarded) features.}
	\label{table:features}
	%\scalebox{0.9}{
	\begin{tabular}{l|llll}
		\hline
		\textbf{Feature} & \textbf{Prefix} & \textbf{Value} & \textbf{Range} & \textbf{Correlations} \\
		\hline
		\sql{order}       & ORD     & frequency  & [0,1]   & \sql{limit}(0.88) \\
		\sql{filter in}   & FIL\_IN & frequency  & [0,16]  & \sql{union}(0.95), FS(0.95) \\			
		\sql{filter}      & FIL     & frequency  & [0,27]  & tp\_???(0.96), TP(0.95) \\
		\sql{count}       & CNT     & frequency  & [0,1]   & \sql{distinct}(1.0) \\
		Triple Patterns   & TP      & frequency  & [1,38]  & \sql{filter}(0.95) \\
		\sql{graph}       & GRA     & frequency  & [0,1]   & - \\
		\sql{optional}    & ORD     & frequency  & [0,9]   & - \\
		\sql{group}       & GRP     & frequency  & [0,4]   & - \\
		(sub)Queries      & Q       & frequency  & [1,10]  & \sql{union}(0.94), \sql{filter in}(0.94) \\
		file size         & FS      & kilobyte  & 1, 10, 100  & \sql{filter in}(0.97), \sql{union}(0.95) \\
		query engine      & -      &  \emph{Vendor}  & -  & - \\
		\hline
	\end{tabular}
	%}
\end{table*}




%

%	\caption{Decision Tree Analysis to identify the reason for query failure, certain error types, and high/low query runtimes. Input for all trees are feature vectors which represent the frequencies of operators and some structural features such as the amount of sub-queries. Also the query engine is added as a categorical feature. Rules in the decision trees are shown in red, sample sizes are encoded as the width of the bottom bar and the value is added inside the bars in bold. For each separate part the class distribution or the average runtime is reported below the bar. \\ 
%		\textbf{Top:} Classification into query success (blue) and failure (red) and incomplete.  
%		The query engine is an important decision rule, which demonstrates that Virtuoso behaves very different from the other systems. For those (left hand side), the amount of sub-queries (Q) and \sql{optional}s play a big role in predicting query failure. \\
%		\textbf{Center:} Classification of query failures into classes incomplete (orange), server error (green), and timeout (purple). \textbf{Gra1} and \textbf{Bla1} only have timeouts. \textbf{ES1} has a large fraction of HTTP errors as well. \\
%		\textbf{Bottom:}  Regression on query runtimes. Red corresponds to high query runtimes, white to low. For \textbf{Vir1} the amount of \sql{filter in} and \sql{group} (GRP) operators are the most important factors. \textbf{Gra1} seems to run slower for queries containing \sql{filter} (FIL) operators. For the other two systems the presence of \sql{filter in} affects the runtimes negatively.
%		\\
%	}




Ontoforce has released the queries for this benchmark run. However, the queries are very complex and sometimes they take up 1 - 100 kb in disk space. To gain a deeper understanding into why queries fail, have timeouts and HTTP errors, why they are fast or slow to execute,... we created a set of features per query and fitted a decision tree~\footnote{\scriptsize \url{http://scikit-learn.org/stable/modules/tree.html}} to the data. The 3 resulting trees are shown in Figure~\ref{fig:Fig10_AllTrees}. The input features for this algorithm are limited by removing highly correlated features. For example \sql{order} and \sql{limit} are highly correlated. The list of retained query features is given in Table~\ref{table:features} together with the highest correlated operators.
By adding `Query Engine' as an additional feature we can train the decision tree on all the available query event data for the Ontoforce Benchmark. 
\begin{itemize}
	\item \textbf{Dominant Feature:} The `Query Engine' is the most important factor to segment the data in all 3 cases. The absence of this feature would in fact indicate that all systems have similar behaviour. \textbf{Vir1} thus is very different: it has fewer errors and query runtimes are significantly smaller.
	\item \textbf{Feature Importance:} If we take the number of node occurrences as a feature as a measure for feature importance then we see 3 features which occur in 5 nodes: TP, \sql{filter in}, \sql{filter}. The \sql{filters} mainly play a role in the decision tree for runtime regression. In predicting failures and error types \sql{optional}, \sql{graph} and Q have the higest occurrences.
	%Fail: Engine:4    Q:1 TP:2, OPT:2, GRA:1, FILIN:1, FIL:0
	%Error: Engine:3   Q:2 TP:1, OPT:1, GRA:1, FILIN:1, FIL:1
	%SUBTOTAL: Engine:7  Q:3 TP:3, OPT:3, GRA:3, FILIN:2, FIL:1, ORD:0, GRP:0  
	%Runtime: Engine:3 Q:1 TP:2, OPT:0, GRA:0, FILIN:3, FIL:4, ORD:1, GRP:1 
	%TOTAL: Engine:10  Q:4 TP:5, OPT:3, GRA:3, FILIN:5, FIL:5, ORD:1, GRP:1  
	\item \textbf{Highest failure rates:} The paths leading to samples with a high failure rate generally contain \sql{optional} operators. All engines except for \textbf{Vir1} suffer when $Q > 1$. \textbf{Gra1} also has a high failure rate for \sql{count} queries.
	\item \textbf{Most frequent error types:} For \textbf{Bla1} and \textbf{Gra1} the errors are all timeouts (purple). For \textbf{ES1} having multiple subqueries often leads to HTTP errors (green). 
	\item \textbf{Queries with high runtimes:} For \textbf{Vir1} and \textbf{ES1} the \sql{filter in} operators are the main cause for high runtimes. For \textbf{Gra1} the presense of \sql{filter}s pushes runtimes above 100s. 
\end{itemize}

Finally we also investigate if the incorrect queries in the \textbf{Vir3} benchmarks had specific query features. Curiously, the problematic queries correspond to the most simple queries: $TP \leq 2$.	




\subsection{Benchmark Cost}
\label{sec:bmcost}


%
%RESULTS Rev1: Notebook Rev1_15 bevat de exacte cijfers.

In this section we aim to get a satellite view on the entire set of benchmarks conducted within this research. The penultimate trade-off for many applications in production is the financial cost for processing a certain workload. Our choice for using cloud hardware and AMIs enables this integrated view on all benchmarks: using cost we can compare single to multi-node setups, the cost for vertical scaling,... 

All financial costs per store and for all benchmarks are shown in Figure~\ref{fig:Fig11_AllSims_Correct}. 
Costs stem from an hourly price for servers on Amazon EC2, together with an hourly license cost for the AMIs.

The instance cost of the AWS hardware was \mbox{\$0.333 /hr} for the 32GB server instances and \$0.667 /hr for the 64GB instances. The licensing costs for the PAGO instances can be found on AWS marketplace and typically scale with the amount of memory per instance. For the 64GB instances, GraphDB's license cost is \$1.4 /hr, for ES \$2 /hr and for Virtuoso \$0.80 /hr. Other systems tested have no licensing cost.

Additionally, before running a benchmark the data has to be ingested in the system. This cost is stacked on top of the query cost in Figure~\ref{fig:Fig11_AllSims_Correct}. For some cases the ingest cost is unimportant as reloading the data is required only rarely.

\begin{itemize}
	\item \textbf{The price of vertical scaling:} Is adding more memory, and therefore a higher license and infrastructure cost a wise choice? If we focus on the \emph{RFI-Optimized} configurations for Watdiv1000M both \textbf{Bla1} and \textbf{Gra1} have lower operational costs when running the higher end hardware. For \textbf{Bla1} the price is lowered from \$27 to \$13.5, for \textbf{Gra1} the reduction is from \$298 to \$230. For the latter mainly the bulk loading process makes it less competitive. For \textbf{Vir1} the price goes from \$5 to \$7.
	
	%LsF 168-> 323 => x2
	%Es  112 -> 475 => x4
	%Vir 5 -> 42
	\item \textbf{The price of horizontal scaling:} As adding more nodes led to higher runtimes, this also translates to higher costs. For \textbf{TPF} the costs go from \$168 to \$323 ($\times 1.9$) , for \textbf{ES} the costs rises from \$112 to \$475 ($\times 4.2$) and for \textbf{Vir} from \$5 to \$42 ($\times 8.4$) 

	\item \textbf{The price for data ingestion:} \textbf{Gra1} seems to have the highest cost for loading the datasets, except for the Ontoforce benchmark. This is interesting as the Ontoforce benchmark has a much bigger dataset (2.4 BT). A possible explanation is that \textbf{Gra1} has trouble ingesting a single gzipped turtle file as was the case for WatDiv, while the Ontoforce dataset was ingested as 42 gzipped N-Quads files.
    For \textbf{Gra1\_64\_Doc} many additional indexes are generated during ingest, which explains the lower cost for \textbf{Gra1\_64\_RFI}. Having more memory by itself can also impact the ingest process, for \textbf{Bla1} the ingest cost is lowered from \$16 to \$12. Virtuoso's bulk loader process is a real trump 
    card in the cost comparisons. The load cost is \$2.8 while \textbf{Bla1} in the optimal case has a cost of \$12.6. The load cost is in fact larger than the runtime cost in this comparison.
    Also for the multi-node setups no advantage is obtained in the ingest phase. \textbf{Vir3} takes 4 times more time to ingest while the cost/hr is also 3 times higher. For \textbf{ES3} a 33\% cost increase is measured, while for \textbf{TPF3} the ingestion becomes 50\% cheaper. The latter however is not \textbf{TPF} specific as the ingestion corresponds to the partitioning and compression of the data with the HDT algorithm (for which we used a 128 GB high-memory infrastructure).

	\item \textbf{The most cost-effective solution:} \textbf{Vir1\_32} is the cheapest solution both for WatDiv1000M as for the Ontoforce benchmark, with costs of respectively \$5 and  \$19. 
\end{itemize}
%SiM						%run2000_cost	load_cost	
%
%Bla_N1_32_W1000_Def		10.440997		16.775985
%Es_N1_32_W1000_Def			90.264271		21.813804
%Gra_N1_32_W1000_Def		45.081634		253.422000
%Vir_N1_32_W1000_Def		2.267796		2.811233
%
%Es_N3_32_W1000_Def			412.789135		63.223079
%Vir_N3_32_W1000_Def		10.366335		32.821250
%
%Bla_N1_64_W1000_Def		2.490707		12.625764
%Es_N1_64_W1000_Def			153.441334		29.685139
%Gra_N1_64_W1000_Def		68.333470		396.480000
%Vir_N1_64_W1000_Def		2.974605		3.411128
%
%Bla_N1_64_W1000_Opt		0.718728		12.862393
%Gra_N1_64_W1000_Opt		7.251239		223.295333
%Vir_N1_64_W1000_Opt		3.288285		3.754633
%
%LdF_N1_64_W1000_Def		162.507679		5.087619
%LdF_N3_64_W1000_Def		319.976409		3.372289
%
%Es_N1_64_Ont_Def			323.740838		45.506300
%Gra_N1_64_Ont_Opt			813.048618		38.892554
%Vir_N1_32_Ont_Opt_VWall	11.172173		8.161853
%Vir_N1_64_Ont_Opt			20.400952		6.107667
%Vir_N3_64_Ont_Opt_0		91.448916		20.090817
 



