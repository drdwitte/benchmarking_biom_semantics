%
%RESULTS Rev1: Notebook Rev1_02  

%exacte cijfers over timeouts en aantal error in data notebook
%Crosstabs (ALL warmup & stress):	Success	Unverified	Incomplete	Timeout Error Unknown (all = 7338)
%Bla_N1_64_Ont_Opt 					36		0			0			431		0	  6871 (94%) 	
%Es_N1_64_Ont_Def 					4246	0			4			1257	1830  ALL 	
%Gra_N1_64_Ont_Opt 					1583	0			0			2372	0	  3383 (46%)	
%Vir_N1_64_Ont_Opt 					7069	120			60			41		48		ALL 	
%Vir32 first test on AWS => 2946 errors! (0 results while consensus > 0) => 41%
%new Vir32 OK same results as Vir64
%Vir64 Incomplete = 6 x 10 queries (<1%) => 1 query incomplete (2^20) andere 0 results terwijl consensus > 0

%Vir3_64_0	many errors but most queries solved at least once => 1122 queries in stres test with at least one susccesful run!
%Vir_N3_64_Ont_Opt_0 				4072	38			27			19		3182		ALL 	
%Vir_N3_64_Ont_Opt_2 				3918	66			2718		0		636			ALL 	
%Reruns telkens maar 100 queries until crash

%TODO notebook contains more info on query correctness

%RESULTS ORIGINAL PAPER
%BMSurvival: 
%	[('Bla_N1_64_Ont_Opt', 55), ('Gra_N1_64_Ont_Opt', 2541), ('ES_N1_64_Ont_Def', 7335), ('Vir_N1_64_Ont_Opt', %7338)]
%ERROR frequency per query
% 						Fail_always	Fail_never 		Fail_sometimes 	Unknown 	thread_type
% 	Bla_N1_64_Ont_Opt 	18 			37 				0 				1168 		warmup
% 	ES_N1_64_Ont_Def 	499 		724 			0 				0 			warmup
% 	Gra_N1_64_Ont_Opt 	233 		988 			0 				2 			warmup
% 	Vir_N1_64_Ont_Opt 	14 			1209 			0 				0 			warmup
%
% 	Bla_N1_64_Ont_Opt 	0 			0 				0 				1223 		stress
% 	ES_N1_64_Ont_Def 	289 		238 			696 			0 			stress
% 	Gra_N1_64_Ont_Opt 	400 		313 			150 			360 		stress
% 	Vir_N1_64_Ont_Opt 	15 			1208 			0 				0 			stress
%GraphDB fails after completing 21\% of the stress test.
%Virtuoso is consistent and successful in 99\% of the queries, ES survived benchmark and is consistently successful for 59\% and 19\% of the queries during %warmup and stress run respectively.
 %[('Vir_N3_64_Ont_Opt_0', 4313), ('Vir_N3_64_Ont_Opt_1', 179), ('Vir_N3_64_Ont_Opt_2', 7338), ('Vir_N3_64_Ont_Opt_AWS1', 116), ('Vir_N3_64_Ont_Opt_AWS2', 224), ('Vir_N3_64_Ont_Opt_AWS3', 228)]



\paragraph{Error Frequencies}
The \emph{Prototypes} and \textbf{Vir*} have been tested on the Ontoforce benchmark for our SWAT4LS~\cite{dewitte_swat4ls_2016} contribution. Note that \textbf{TPF} systems do not currently support all SPARQL operators and could therefore not be run on this benchmark.  In Figure~\ref{fig:Fig09_FailuresOntoforceBM} we show the results for the \emph{Vendor} systems. Each simulation consists of a small bar, corresponding to the single-threaded warm-up run, and 5 concatenated bars corresponding to 5 threads in the stress test.
The Figure also shows that only Virtuoso simulations had a sufficiently wide benchmark survival interval to enable further analysis.

\begin{itemize}
	\item \textbf{Bla1\_64\_RFI:} One major difference with the results on the WatDiv benchmark is Blazegraph's inability to handle the complexity of the Ontoforce queries, resulting in very short benchmark survival interval: it contains only 55 queries, of which 18 are timeouts.
	
	\item \textbf{Gra1\_64\_RFI:} GraphDB also did not survive the entire benchmark, but managed to stay up for 21\% of the stress run. During the stress run it solved 40\% of the queries successfully, the other queries resulted in a timeout. For 38\% of the queries, at least one successful run is available in the stress run.

	\item \textbf{ES1\_64\_Doc:} ES was definitely the least successful on the WatDiv benchmarks, but is the only store, apart from Virtuoso, for which the benchmark survival interval spans the entire benchmark. 58\% of the queries were executed successfully. The remainder consists of 25\% HTTP errors and 17\% timeouts. 
%%Vir_N1_64_Ont_Opt 	7069	120			60			41		48		ALL 		
	\item \textbf{Vir1\_*\_RFI:} Virtuoso is both consistent and successful on this benchmark with only 1\% of queries consistently failing, overall the success rate is 98\%. These failures correspond mainly to queries which contain \emph{property paths}. None of the other stores could handle these queries. 
	It should be noted that during the creation of the DISQOVER product, Virtuoso was frequently used as a back-end system, which partially implies a certain favorable bias in the Ontoforce results.The \textbf{Vir1\_32\_RFI} in the SWAT4LS~\cite{dewitte_swat4ls_2016} paper had 41\% incomplete queries. This re-run however, achieves the same figures as the 64GB run.
	
	\item \textbf{Vir3\_64\_RFI\_*:} The  \textbf{Vir3\_64\_RFI} setup was re-run multiple times, the different runs are identified with an additional sequence number 0-2.
	Although the success rate of \textbf{Vir3\_64\_RFI\_0} is only 55\%, 92\% of the queries are successfully executed at least once, which makes it possible to make runtime comparisons. \textbf{Vir3\_64\_RFI\_2} has far less reported errors. Post-processing revealed issues with query completeness (orange) for 37\% of the queries.
\end{itemize}

 
\paragraph{Query Correctness.}

%\todo{Query Correctness sectie maar ook eentje over Query completeness}
Previously published results~\cite{dewitte_swat4ls_2016} had counter-intuitive runtime results:  \textbf{ Vir1\_32} and \textbf{Vir3\_64\_RFI\_2} executed much faster than \textbf{Vir1\_64}. 
Consequently, we studied the number of results per query:

 
\begin{itemize}
	\item \textbf{Inter-thread consistency:} As a first step we analyzed whether for each individual system the number of query results was consistent for each query-mix. 
	Without any exception this inter-thread consistency was confirmed.
	
	\item \textbf{Query consensus:} In the query event format, described in Table~\ref{table:queryevents}, one field indicates whether a query is correct or its result count incomplete. These values are obtained by creating a query consensus, with the following rules. If at least two separate \emph{Vendor} systems agreed on the number of results we assume this results is `correct', for 97.3\% this is the case. If only 1 engine can solve a query we label these as `uncertain'. Virtuoso solves 19 queries for which no consensus can be derived. For 13 queries none of the systems managed to generate a solution. 8 of these contain a property path operator, the other 5 have \sql{FILTER IN} operators containing large URI lists, such that the file size of the query is between 10 and 100 kB.
	
	\item \textbf{Count Queries:} Of the 19 `uncertain' queries solved by Virtuoso 15 are \sql{count} queries. However, upon inspection the \sql{count} operator was always part of a sub-query, so this result can not be disproven. The benchmark software only reports the number of results per query. We extended it to also download the actual results to be able to verify whether the \sql{count} queries are consistent between the stores. However, no inconsistencies were found there.
	
	\item \textbf{Incorrect Query results:} Some of the Virtuoso benchmarks have incorrect results. The typical pattern is that the query is executed $< 1s$ and generates 0 results. 1 query also had the query result limit $ = 2^{20}$. To get more insight into the context of incomplete queries we executed the \textbf{Vir3\_64} benchmark an additional three times. In these runs the incorrect query results were not observed, but the new benchmarks never made it to the stress test, with the best run having a benchmark survival interval with a length of 228 queries.
\end{itemize}




